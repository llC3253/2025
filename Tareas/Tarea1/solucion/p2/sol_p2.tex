\documentclass[11pt]{article}


\usepackage[utf8]{inputenc}
\usepackage{fullpage}
\usepackage{epsfig}
\usepackage{amsmath}
\usepackage{amssymb}
\usepackage{multicol}
\usepackage{color}
\usepackage{hyperref}
\usepackage{xcolor}
\usepackage{dirtree}
\usepackage{fontawesome}
\usepackage{tikz}


\usetikzlibrary{trees}


\newcommand{\comm}[1]{{\bf {\color{red} #1}}}
\newcommand{\Enc}{\textit{Enc}}
\newcommand{\Dec}{\textit{Dec}}
\newcommand{\Gen}{\textit{Gen}}

\newcommand{\M}{\mathcal{M}}
\newcommand{\C}{\mathcal{C}}
\newcommand{\K}{\mathcal{K}}

\begin{document}

\begin{center}
  \bf Criptografía y Seguridad Computacional - IIC3253\\
  \bf Tarea 1\\
  \bf Solución pregunta 2
\end{center}

\bigskip

\noindent
Sean $q$ y $n$ dos números naturales tales que $6 \leq q \leq n$, y
sea $(\textit{Gen}, \textit{Enc}, \textit{Dec})$ un esquema
criptográfico definido sobre los espacios $\mathcal{M} = \mathcal{C} =
\{0,1\}^n$ y $\mathcal{K} = \{0, 1\}^{nq - 1}$. En esta pregunta usted
debe demostrar que este esquema no es una pseudo-random permutation
(PRP) con $q$ rondas. En particular, debe demostrar que el adversario
gana el juego que define una PRP con una probabilidad mayor o igual a
$\frac{1}{2} + \frac{1}{6}$.

\bigskip

\noindent
{\bf Solución.} Sea $(\textit{Gen}, \textit{Enc}, \textit{Dec})$ un
esquema criptográfico arbitrario definido sobre los espacios
$\mathcal{M} = \mathcal{C} = \{0,1\}^n$ y $\mathcal{K} = \{0, 1\}^{nq
  - 1}$. Suponemos que este esquema es perfectamente correcto, lo cual
implica que para cada $k \in \{0, 1\}^{nq - 1}$, la función
$\textit{Enc}_k : \{0,1\}^n \to \{0,1\}^n$ es una permutación (es
decir, es una función biyectiva).

Considere la definición de pseudo-random permutation (PRP) dada en
clases. Para demostrar que el esquema anterior no es una PRP con $q$
rondas, usamos un adversario que ejecuta los sigui\-entes~pasos:
\begin{itemize}
  \item El adversario entrega $z_1$, $\ldots$, $z_q$ al verificador,
    donde $z_i \in \{0,1\}^n$ para cada $i \in \{1, \ldots, q\}$ y
    $z_i \neq z_j$ para cada $i,j \in \{1, \ldots, q\}$ tal que $i \neq
    j$. El adversario recibe como respuesta $f(z_1)$,
    $\ldots$, $f(z_k)$.

  \item El adversario verifica se existe $k \in \{0, 1\}^{nq - 1}$
    tal que $f(z_i) = \textit{Enc}_{k}(z_i)$ para cada $i \in \{1,
    \ldots, q\}$.  Si esta condición se cumple, entonces el adversario
    indica que $b=0$, sino indica que $b=1$.
\end{itemize}
Note que para implementar el segundo paso se deben considerar
$2^{nq-1}$ claves en el peor de los casos. Vale decir, el
adversario es un algoritmo de tiempo exponencial, lo cual no es un
problema puesto que la definición de PRP no impone restricciones sobre
su tiempo de funcionamiento.

Necesitamos calcular la probabilidad de que el adversario gane el
juego, lo cual está dado por la siguiente expresión:
\begin{align*}
  \Pr(\text{Adv}&\text{ersario gane el juego}) \ =\\
  &\Pr(\text{Adversario gane el juego} \mid b=0) \cdot \Pr(b=0) \ +\\
  &\hspace{162pt} \Pr(\text{Adversario gane el juego} \mid b=1) \cdot \Pr(b=1) \ =\\
  &\frac{1}{2} \cdot \Pr(\text{Adversario gane el juego} \mid b=0) + \frac{1}{2} \cdot \Pr(\text{Adversario gane el juego} \mid b=1).
\end{align*}
Si $b = 0$, entonces el verificador debe haber encriptado los mensaje
$z_1$, $\ldots$, $z_q$ con una clave $k \in \{0,1\}^{nq-1}$. Tenemos
entonces que el adversario elige $b = 0$ y gana el juego. Se concluye
que $ \Pr(\text{Adversario gane el juego} \mid b=0) = 1$.  Si $b = 1$,
entonces el verificador escoge al azar una permutación $\pi :
\{0,1\}^n \to \{0,1\}^n$ y responde en el juego con el valor $f(z) =
\pi(z)$. En este caso, tenemos que el adversario pierde el juego si
existe $k \in \{0,1\}^{nq-1}$ tal que $f(z_i) = \textit{Enc}_k(z_i)$
para cada $i \in \{1, \ldots, q\}$. Vale decir,
\begin{eqnarray*}
  \Pr(\text{Adversario pierda el juego} \mid b=1) &=&
  \Pr_{\pi}\bigg(\bigvee_{k \in \{0,1\}^{nq-1}} \bigwedge_{i=1}^q \pi(z_i) = \textit{Enc}_k(z_i)\bigg)\\
  & \leq &  \sum_{k \in \{0,1\}^{nq-1}} \Pr_{\pi}\bigg(\bigwedge_{i=1}^q \pi(z_i) = \textit{Enc}_k(z_i)\bigg)\\
  &=& \sum_{k \in \{0,1\}^{nq-1}} \frac{(2^n - q)!}{(2^n)!}\\
  &=& \frac{2^{nq - 1}}{{\displaystyle \prod_{i=0}^{q-1} (2^n - i)}}.
\end{eqnarray*}
Sea $f$ la función definida como $f(n, q) = \frac{2^{nq -
    1}}{\prod_{i=0}^{q-1} (2^n - i)}$. Es posible
demostrar que $f(6, 6)$ es el mayor valor de esta función en el
conjunto de puntos $\{ (n,q) \in \mathbb{N} \times \mathbb{N} \mid 6 \leq q \leq n\}$. Calculando el
valor de $f(6, 6)$, concluimos que:
\begin{eqnarray*}
  \Pr(\text{Adversario pierda el juego} \mid b=1) \ \ \leq \ \ f(6, 6) \ \ < \ \ \frac{2}{3}.
\end{eqnarray*}
Tenemos entonces que $\Pr(\text{Adversario gane el juego} \mid b=1) \geq \frac{1}{3}$, de lo cual se concluye que:
\begin{align*}
  \Pr(\text{Adv}&\text{ersario gane el juego}) \ =\\
  &\frac{1}{2} \cdot \Pr(\text{Adversario gane el juego} \mid b=0) + \frac{1}{2} \cdot \Pr(\text{Adversario gane el juego} \mid b=1) \ \geq\\
  &\frac{1}{2} \cdot 1 + \frac{1}{2} \cdot \frac{1}{3} \ = \ \frac{1}{2} + \frac{1}{6},
\end{align*}
que era lo que queríamos demostrar.
  
\end{document}
