%!TEX root = ../main/main.tex

Sean $q$ y $n$ dos números naturales tales que $6 \leq q \leq n$, y sea $(\textit{Gen}, \textit{Enc}, \textit{Dec})$ un esquema criptográfico definido sobre los espacios $\mathcal{M} = \mathcal{C} = \{0,1\}^n$ y $\mathcal{K} = \{0, 1\}^{nq - 1}$. En esta pregunta usted debe demostrar que este esquema no es una pseudo-random permutation (PRP) con $q$ rondas. En particular, debe demostrar que el adversario gana el juego que define una PRP con una probabilidad mayor o igual a $\frac{1}{2} + \frac{1}{6}$.

\paragraph{Corrección.}
Esta pregunta se corrige considerando que se debe definir la
estrategia del adversario que le permite ganar el juego de $q$ rondas
que define una PRP con una probabilidad mayor o igual a $\frac{1}{2}
+ \frac{1}{6}$. La asignación de puntaje en esta pregunta es la
siguiente.
\begin{itemize}
    \item{[1.5 puntos]} Sólo se entrega la definición de la estrategia
    del adversario. Esta estrategía tiene elementos que efectivamente
    podrían permitir al adversario ganar con una probabilidad
    significativamente mayor a $\frac{1}{2}$, pero tiene algunos
    errores.

    \item{[3 puntos]} Se entrega una definición de la estrategia del
    adversario que está correcta en el sentido de que le permite al
    adversario ganar el juego con probabilidad mayor o igual a
    $\frac{1}{2} + \frac{1}{6}$.

    \item{[4.5 puntos]} Se entrega una definición de la estrategia del
    adversario que está correcta en el sentido de que le permite al
    adversario ganar el juego con probabilidad mayor o igual a
    $\frac{1}{2} + \frac{1}{6}$. Además se explica por qué la
    probabilidad de que el adversario gane el juego es mayor o igual a
    $\frac{1}{2} + \frac{1}{6}$, pero no se entrega una demostración
    completa de esta propiedad.


    \item{[6 puntos]} Se entrega una definición de la estrategia del
    adversario que está correcta en el sentido de que le permite al
    adversario ganar el juego con probabilidad mayor o igual a
    $\frac{1}{2} + \frac{1}{6}$. Además se entrega una demostración
    formal de que la probabilidad de que el adversario gane el juego
    es mayor o igual a $\frac{1}{2} + \frac{1}{6}$.
\end{itemize}

