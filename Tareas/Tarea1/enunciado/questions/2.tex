
Sean $q$ y $n$ dos números naturales tales que $6 \leq q \leq n$, y sea $(\textit{Gen}, \textit{Enc}, \textit{Dec})$ un esquema criptográfico definido sobre los espacios $\mathcal{M} = \mathcal{C} = \{0,1\}^n$ y $\mathcal{K} = \{0, 1\}^{nq - 1}$. En esta pregunta usted debe demostrar que este esquema no es una pseudo-random permutation (PRP) con $q$ rondas. En particular, debe demostrar que el adversario gana el juego que define una PRP con una probabilidad mayor o igual a $\frac{1}{2} + \frac{1}{6}$.
