%!TEX root = ../main/main.tex

Sea $\{h^n\}_{n \in \mathbb{N}}$ una familia de funciones de
compresión resistente a colisiones tal que $h^n
: \{0,1\}^{2n} \to \{0,1\}^{n}$. Supondremos también que esta
familia es \textit{puzzle friendly}, lo que significa que no existe un
algoritmo eficiente que es capaz de encontrar una palabra $x$ que resuelve un puzzle
$h^n(u \| x) = v$. Formalmente, un puzzle es un par $(u, v) \in \{0,1\}^{2n}$, y una solución para el puzzle es una palabra $x \in \{0, 1\}^n$ tal que $h(u \| x) = v$. Si existe tal $x$, se dice que el puzzle $(u,v)$ tiene solución. Con esta notación, la familia
$\{h^n\}_{n \in \mathbb{N}}$ se dice \textit{puzzle friendly} si para cada
 algoritmo aleatorizado $\mathcal{A}$ tal que:
\begin{itemize}
\item con entrada $(u, v) \in \{0,1\}^{2n}$, el algoritmo $\mathcal{A}$ retorna una palabra $\mathcal{A}(u, v) \in \{0,1\}^n$ o el símbolo $\bot$, 

\item $\mathcal{A}$ realiza $o(n \cdot 2^n$) operaciones\footnote{Recuerde que una función $f(n)$ es $o(g(n))$ si se cumple que $(\forall c \in \mathbb{R}, c > 0)(\exists n_0 \in \mathbb{N})(\forall n \in \mathbb{N}, n \geq n_0)(f(n) \leq c \cdot g(n))$.} para cada entrada $u, v \in \{0,1\}^n$,
\end{itemize}
se tiene que la siguiente función de $n$ es despreciable:
\begin{eqnarray*}
\max_{v \in \{0,1\}^n} \Pr_{u \sim \mathbb{U}(\{0,1\}^n)}\big[\mathcal{A} \text{ soluciona el puzzle } (u,v)\big],
\end{eqnarray*}
donde $u \sim \mathbb{U}(\{0,1\}^n)$ indica que $u$ es escogido al azar con distribución uniforme desde el conjunto $\{0,1\}^n$, y $\mathcal{A}$ soluciona el puzzle $(u,v)$ si $\mathcal{A}(u,v) \in \{0,1\}^n$ y $h(u \| \mathcal{A}(u,v)) = v$ en caso de que el puzzle $(u,v)$ tenga solución, y $\mathcal{A}(u,v) = \bot$ en caso de que el puzzle $(u,v)$ no tenga solución.

A partir de la familia $\{h^n\}_{n \in \mathbb{N}}$, definimos el
protocolo {\bf EstablecerClave}($1^n$) que permite a dos usuarios $A$ y $B$
establecer una clave de $n$ bits en un canal público, sin la necesidad de juntarse físicamente.
\begin{flushleft}
{\bf EstablecerClave}($1^n$)
\begin{enumerate}
\item[(1)] $A$ escoge con distribución uniforme $s \in \{0,1\}^n$, y se lo envía a $B$.

\item[(2)] \begin{minipage}[t]{\linewidth}
Sea $P$ el conjunto de las primeras $n^2$ palabras en $\{0,1\}^n$, ordenadas por orden lexicográfico (definido por $0 < 1$), y sea $m = n \cdot \lceil \log n \rceil$. Por ejemplo, si $n = 5$, entonces $P=\{00000$, $00001$, $00010$, $\ldots$, $10110$, $10111$, $11000 \}$ y $m = 15$.
\end{minipage}
\begin{enumerate}
\item[(2.1)] \begin{minipage}[t]{\linewidth} $A$ escoge $m$ palabras distintas $x_1$, $\ldots$, $x_m$ desde el conjunto $P$, calcula $a_i = h(s \| x_i)$ para cada $i \in \{1, \ldots, m\}$, y envía $(1,a_1)$, $\ldots$, $(m,a_m)$ a $B$.
\end{minipage}

\item[(2.2)] $B$ escoge $m$ palabras distintas $y_1$, $\ldots$, $y_m$ desde el conjunto $P$, calcula $b_i = h(s \| y_i)$ para cada $i \in \{1, \ldots, m\}$, y envía $(1, b_1)$, $\ldots$, $(m, b_m)$ a $A$.
\end{enumerate}
\item[(3)] \begin{minipage}[t]{\linewidth}
Sea $I = \{ (i,j) \mid a_i = b_j\}$. Si $I = \emptyset$, entonces el protocolo falla. En otro caso, sea $(k, \ell)$ el menor elemento en $I$ en el orden lexicográfico sobre $\{1, \ldots, m\}^2$ definido por $1 < 2 < \dots < m$.
\end{minipage}
\begin{enumerate}
\item[(3.1)] $A$ establece como clave $x_{k}$.
\item[(3.2)] $B$ establece como clave $y_\ell$ (que debería ser igual a $x_k$).
\end{enumerate}
\end{enumerate}
\end{flushleft}
En esta pregunta usted va a analizar la complejidad de este protocolo,
para lo cual va a considerar las operaciones entre bits (suma, resta,
comparación, etc.) como las operaciones básicas en los algoritmos, las
cuales tienen costo 1. Por ejemplo, verificar si $u = v$ para dos
palabras $u, v \in \{0,1\}^n$ toma tiempo $n$ ya que se deben realizar
$n$ operaciones de comparación entre bits. En el análisis a realizar a
continuación debe suponer que $h^n(u \| v)$ se calcula en tiempo
$O(n)$, lo cual es cierto para las funciones de hash usuales.
\begin{enumerate}
\item La llamada {\bf EstablecerClave}($1^n$) falla tanto si no se tiene un par $(i, j)$ tal que $a_i = b_j$ como si $x_k \neq y_\ell$ (las claves secretas establecidas por $A$ y $B$ son distintas). Demuestre que existe una función despreciable $f(n)$ tal que:
\begin{eqnarray*}
\Pr(\text{{\bf EstablecerClave}($1^n$) falle}) & \leq & f(n).
\end{eqnarray*}

\item Suponga que {\bf EstablecerClave} no falla. Demuestre que $A$ y $B$ establecen una clave compartida en tiempo $O(n^2 \cdot \log^2 n)$.

\item Suponga que {\bf EstablecerClave} no falla, y que un atacante trata de descubrir la clave compartida entre $A$ y $B$. Suponga que el atacante es exitoso en el sentido de que logra construir un algoritmo (no aleatorizado) $\mathcal{A}$ que dado $s, u_1, u_2, v \in \{0,1\}^n$ tal que $|\{ u \in \{0, 1\}^n \mid u_1 \leq u \leq u_2 \}| = n^2$, genera $u \in \{0,1\}^n$ que satisface $h(s \| u) = v$ y $u_1 \leq u \leq u_2$ siempre que dicho $u$ exista, y retorna $\bot$ si dicho $u$ no existe. Para la construcción anterior $\mathcal{A}$ realiza $o(n^3)$ operaciones, donde $\leq$ es el orden lexicográfico sobre $\{0,1\}^n$ definido por $0 < 1$. Demuestre que esto lleva a una contradicción puesto que implicaría que la familia de funciones $\{h^n\}_{n \in \mathbb{N}}$ no es puzzle friendly.


\end{enumerate}




